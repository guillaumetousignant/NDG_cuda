\chapter{Introduction}
Fluid flows are an important part of everyday life. Fluid mechanics is a whole discipline dedicated
to studying these flows. More and more industrial and scientific applications surface everyday, from
great scales like climate studies encompassing the whole earth over millennia, to the study of
microscopic cells flowing through tiny blood vessels. The aerospace industry is probably the most
evident example of how prevalent aerodynamics are in today's world. Before the advent of computers,
there was really only two methods used in engineering fluid dynamics: experimental and theoretical.
Since the widespread use of computers became the norm, a third method appeared:
\textit{computational fluid dynamics (CFD)}. This is a numerical approach to solving the equations
governing those processes.

Usage of CFD has been steadily rising in recent years~\cite{Slotnick2014}. It is easy to imagine
why, when experimental results are so costly to obtain, and a theoretical approach is difficult to
apply to complex cases. Nonetheless, CFD is not meant to replace those methods, but to be combined
with them. Reducing the need to perform experiments, while comparing results with the experiments
that do happen to verify and validate them~\cite{Stern2001}.