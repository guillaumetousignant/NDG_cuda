\chapter{Introduction}

Fluid flows are an important part of everyday life. Fluid mechanics is a whole discipline dedicated
to studying these flows. More and more industrial and scientific applications surface everyday, from
great scales like climate studies encompassing the whole earth over millennia, to the study of
microscopic cells flowing through tiny blood vessels. The aerospace industry is probably the most
evident example of how prevalent aerodynamics are in today's world. Before the advent of computers,
there was really only two methods used in engineering fluid dynamics: experimental and theoretical.
Since the widespread use of computers became the norm, a third method appeared:
\textit{\acrfull{acr:CFD}}. This is a numerical approach to solving the equations governing those
processes.

Usage of \acrshort{acr:CFD} has been steadily rising in recent years~\cite{Slotnick2014}. It is easy
to imagine why, when experimental results are so costly to obtain, and a theoretical approach is
difficult to apply to complex cases. Nonetheless, \acrshort{acr:CFD} is not meant to replace those
methods, but to be combined with them. Reducing the need to perform experiments, while comparing
results with the experiments that do happen to verify and validate them~\cite{Stern2001}.

Even with increasing computing power, some problems are difficult to solve with \acrshort{acr:CFD}.
Problems of high-dimensionality and complex flows still cannot be solved economically. NASA's
\acrshort{acr:CFD} Vision 2030 study~\cite{Slotnick2014} states that turbulent flow separation is
one area that still cannot be predicted accurately. In order to simulate these problems, we will
need increased processing power, more efficient use of that power, and higher resolution numerical
methods.

% DG-SEM
Spectral methods are an answer to that need of more accurate numerical methods. Unlike traditional
methods like finite differences methods, finite element methods and finite volume methods, spectral
methods are high-order methods and converge exponentially fast. Instead of modelling the solution as
single points, or a single to double order function within a space subdivision, spectral methods
represent a function as a truncated series:

\begin{equation}
	u(x) \approx u_N(x) = \sum_{n = 0}^{N} \widehat{u}_n \phi _n(x)
\end{equation}

Where $\phi _n(x)$ are basis functions. We use this representation into the governing equations to
compute the unknowns $\widehat{u}_n$. Spectral methods are usually divided in two
categories~\cite{Karniadakis2005}: collocation methods and Galerkin methods. Galerkin methods solve
governing equations in integral form over the domain~\cite{Reed1973}, which is broken up into
elements. This method, in combination with the spectral approximation, will be used in this work. It
is called the \textit{discontinuous Galerkin spectral element method (DG-SEM)}. The method is
discontinuous because it allows the solution to be discontinuous at element boundaries, instead
using fluxes between elements to stabilise the scheme. This method combines the accuracy and
exponential convergence of spectral methods with the geometric flexibility of finite element
methods. 

% GPUs
With the stagnation of traditional computer \textit{central processing units (CPU)} operating
frequencies over the last years~\cite{Parkhurst2006}, computer systems have evolved to use more
parallel architectures. CPUs are now composed of several computing cores~\cite{Nayfeh1997} executing
tasks together or in parallel, and contemporary \textit{\acrfull{acr:HPC}} platforms consist of
several whole computers networked together executing tasks in parallel. Amidst those changes, an
inherently parallel architecture has started to be used in scientific computing. \textit{Graphics
processing units (GPU)} are computer chips that were initially used in computer graphics, a
massively parallel workload. GPUs can now be programmed similarly to traditional
CPUs~\cite{Owens2008}, offering their thousands of simpler cores to many kids of computation. This
architecture is optimised for maximal bandwidth, to process great amounts of data as fast as
possible. Recently, these processors have been incorporated into \acrshort{acr:HPC}
platforms~\cite{Fan2004}, enabling massively parallel workloads with theoretical processing power
much greater than using traditional CPUs.

We will use GPUs for our computations in hope to increase the available processing power to solve
complex problems. This will not come free, as GPU architectures are more optimised for static
workloads executing the exact same code on all GPU cores. We use the \acrshort{acr:CUDA} parallel
computing platform~\cite{Garland2008}, which enables programming GPUs using the C++ language with
some extensions, much like CPU programming.

% Adaptivity
Even with highly accurate methods and the processing power of GPU-enables \acrshort{acr:HPC}
platforms, the available resources must be spent judiciously. To get the accurate results we want,
meshes need to be extremely fine. A mesh this fine throughout the whole domain can become
impractical to compute even on the largest \acrshort{acr:HPC} systems. The limited resources must be
spent where they will count most. Creating grids that are more refined in areas of interest is
possible if those areas are known beforehand. This is a time-consuming process that must be started
over for every new problem. It is sometimes not possible to predict where areas of interest will be,
such as when modeling chaotic turbulent flows. It is also possible that the mathematically important
areas are not easy to identify. 

\textit{\Acrfull{acr:AMR}} methods have been studied in order to solve this problem. These methods
aim to identify the areas of the mesh that need refinement, and refine those areas to obtain more
accurate results while not wasting resources on areas of less interest. ~\cite{Berger1984}

% Load balancing

% Problem