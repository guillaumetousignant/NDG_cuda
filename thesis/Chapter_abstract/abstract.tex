\thispagestyle{plain} % stop the headers being added in

\begin{center}
    \vspace*{0cm} % vertical gap. *" makes sure Latex does not ignore the command. 
    \phantomsection\addcontentsline{toc}{chapter}{Abstract}
    {\large
        \textbf{A Graphics Processing Unit Based \\ 
            Discontinuous Galerkin Wave Equation Solver \\
            with hp-Adaptivity and Load Balancing \\
        }
    }
    \vspace{0cm}
    \normalsize

    by \\
    \vspace{0cm}
    \textbf{Guillaume Tousignant}

    \vspace{0.4cm}
    \large
    \textbf{Abstract}
\end{center}

\begin{adjustwidth}{-0.5in}{-0.5in}

\hspace{\parindent} % Don't know why it isn't indented by default
In \acrlong{acr:CFD}, we often need to solve complex problems with high precision and efficiency. We
propose a three-pronged approach to attain this goal. First, we use the
\textit{\acrfull{acr:DG-SEM}} for its high accuracy. Second, we use \textit{\acrfullpl{acr:GPU}} to
perform our computations to exploit available parallel computing power. Third, we implement a
parallel \textit{\acrfull{acr:AMR}} algorithm to efficiently use our computing power where it is
most needed. We present a \acrshort{acr:GPU} \acrshort{acr:DG-SEM} solver with \acrshort{acr:AMR}
and dynamic load balancing for the 2D wave equation. 

The \acrshort{acr:DG-SEM} is a higher-order method that splits a domain into elements and represents
the solution within these elements as a truncated series of orthogonal polynomials. This approach
combines the geometric flexibility of finite-element methods with the exponential convergence of
spectral methods.

\Acrshortpl{acr:GPU} provide a massively parallel architecture, achieving a higher throughput than
traditional \acrshortpl{acr:CPU}. They are relatively new as a platform in the scientific community,
therefore most algorithms need to be adapted to that new architecture. We perform most of our
computations in parallel on multiple \acrshortpl{acr:GPU}.

\Acrshort{acr:AMR} selectively refines elements in the domain where the error is estimated to be
higher than a prescribed tolerance, via two mechanisms: \textit{p-refinement} increases the
polynomial order within elements, and \textit{h-refinement} splits elements into several smaller
ones. This provides a higher accuracy in important flow regions and increases capabilities of
modeling complex flows, while saving computing power in other parts of the domain. We use the
\textit{mortar element method} to retain the exponential convergence of high-order methods at the
non-conforming interfaces created by \acrshort{acr:AMR}.

We implement a parallel \textit{dynamic load balancing} algorithm to even out the load imbalance
caused by solving problems in parallel over multiple \acrshortpl{acr:GPU} with \acrshort{acr:AMR}.
We implement a \textit{\acrlong{acr:SFC}}-based repartitioning algorithm which ensures good locality
and small interfaces.

While the intense calculations of the high order approach suit the \acrshort{acr:GPU} architecture,
programming of the highly dynamic adaptive algorithm on \acrshortpl{acr:GPU} is the most challenging
aspect of this work. The resulting solver is tested on up to \(64\) \acrshortpl{acr:GPU} on
\acrshort{acr:HPC} platforms, where it shows good strong and weak scaling characteristics. Several
example problems of increasing complexity are performed, showing a reduction in computation time of
up to \(3 \times \) on \acrshortpl{acr:GPU} vs \acrshortpl{acr:CPU}, depending on the loading of the
\acrshortpl{acr:GPU} and other user-defined choices of parameters. \Acrshort{acr:AMR} is shown to
improve computation times by an order of magnitude or more.

\begin{footnotesize}
\begin{spacing}{1.125}
\vspace*{\fill}
\noindent
\textbf{Keywords}: \Acrlongpl{acr:SEM}, discontinuous Galerkin, \acrlongpl{acr:GPU},
\acrlong{acr:AMR}, \acrlongpl{acr:SFC}, Hilbert curve, dynamic load balancing, \acrlong{acr:HPC}.
\end{spacing}
\end{footnotesize}

\end{adjustwidth}
