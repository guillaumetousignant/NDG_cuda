\thispagestyle{plain} % stop the headers being added in

\begin{center}
	\vspace*{0.4cm} % vertical gap. *" makes sure Latex does not ignore the command. 
	\phantomsection\addcontentsline{toc}{chapter}{Abstract}
	{ \Large
		\textbf{A Graphics Processing Unit Based \\ 
			Discontinuous Galerkin Wave Equation Solver \\
			with hp-Adaptivity and Load Balancing \\
		}
	}
	\vspace{0.3cm}
	\large

	by \\
	\vspace{0.3cm}
	\textbf{Guillaume Tousignant}
	
	\vspace{0.8cm}
	\textbf{Abstract}
    \vspace{0.25cm}
\end{center}

\begin{adjustwidth}{-0.5in}{-0.5in}

\hspace{\parindent} % Don't know why it isn't indented by default
In \acrfull{acr:CFD}, we often need to solve complex problems with high precision. We propose a
three-pronged approach to attain this goal. First, we use the \textit{\acrfull{acr:DG-SEM}} for its
high accuracy. Second, we use \textit{\acrfullpl{acr:GPU}} to perform our computations to increase
the available parallel computing power. Third, we implement a parallel \textit{\acrfull{acr:AMR}}
algorithm to efficiently use our computing power where it is most needed. We present a
\acrshort{acr:GPU} \acrshort{acr:DG-SEM} solver with \acrshort{acr:AMR} and dynamic load balancing
for the 2D wave equation. 

The \acrshort{acr:DG-SEM} is a higher-order method that splits a domain into elements and represents
the solution within these elements as a truncated series of orthogonal polynomials. This approach
combines the geometric flexibility of finite-element methods with the exponential convergence of
spectral methods.

\Acrshortpl{acr:GPU} have a massively parallel architecture, achieving a higher throughput than
traditional \acrshortpl{acr:CPU}. They are relatively new as a platform in the scientific community,
therefore most algorithms need to be adapted to that new architecture. We perform most of our
computations in parallel on multiple \acrshortpl{acr:GPU}.

\Acrshort{acr:AMR} selectively refines elements in the domain where the error is estimated to be
higher, via two mechanisms: \textit{p-refinement} increases the polynomial order within elements,
and \textit{h-refinement} splits elements into several smaller ones. This provides a higher accuracy
in important flow regions and enables modeling complex flows, while saving computing power in other
parts of the domain. We use the \textit{mortar element method} to retain the exponential convergence
of high-order methods at the non-conforming interfaces created by \acrshort{acr:AMR}.

We implement a parallel \textit{dynamic load balancing} algorithm to even out the load imbalance
caused by solving problems in parallel over multiple \acrshortpl{acr:GPU} with \acrshort{acr:AMR}.
We implement a \textit{\acrfull{acr:SFC}}-based repartitioning algorithm for their good locality and
small interfaces.

Using \acrshortpl{acr:GPU} on a \acrshort{acr:HPC} platform, are able to obtain a \(3 \times \)
faster execution time than using \acrshortpl{acr:CPU} for the core solver. \Acrshort{acr:AMR} is
shown to improve computation times by \(67 \times \) compared to using a uniformly completely
refined mesh.

\noindent
\textbf{Keywords}: \Acrlongpl{acr:SEM}, discontinuous Galerkin, \acrlongpl{acr:GPU}, \acrlong{acr:AMR}, \acrlongpl{acr:SFC}, Hilbert curve, dynamic load balancing, \acrlong{acr:HPC}.

\end{adjustwidth}
