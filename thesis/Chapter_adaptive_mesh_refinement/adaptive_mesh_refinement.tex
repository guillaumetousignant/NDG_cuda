\chapter{Adaptive Mesh Refinement} \label{chapter:adaptive_mesh_refinement} 
% Something about move semantics
% No coarsening
% Adaptivity triggers
% Mortar here

Computers constantly increase in power, thanks to incremental progress made on known processes and
new architectures such as that described in chapter~\ref{chapter:graphics_processing_units}.
However, processing power and memory is still limited and the size of problems studied has
increased in step with the available ressources. It is still necessary to carefully manage those
limited ressources in order to maximise the efficiency of simulations. Some flow regions may be more
interesting or harder to compute, benefiting from an increase in resolution. On the other hand, some
flow regions may have less happening in them or be easier to compute, and a decrease in resolution
may be acceptable.

It is possible to increase the number of elements and/or the polynomial order of an entire mesh
before solving the problem. This increases resolution in important areas of the flow, but also
increases resolution everywhere else in the domain, where the increased computation cost provides no
benefit. It is sometimes possible to predict where to refine before solving the problem, such as
around static shock waves in predictable locations. In these cases, the mesh can be refined in those
areas before computation has started. However, it is not always possible to know these areas
beforehand, or these areas may move as time advances if the problem is transient. The error may also
be higher in unforeseen areas that are not apparent.

Adaptive mesh refinement is the process of mesh resolution as the computation goes, where the effect
of that increase in resolution is most needed. 

\section{Strategies} \label{section:adaptive_mesh_refinement:adaptivity_strategies}

\section{Error Estimation} \label{section:adaptive_mesh_refinement:error_estimation}

\section{Refinement Criteria} \label{section:adaptive_mesh_refinement:refinement_criteria}

\section{Mortar Element Method} \label{section:adaptive_mesh_refinement:mortar_element_method}

\section{Implementation} \label{section:adaptive_mesh_refinement:implementation}