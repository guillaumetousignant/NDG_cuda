\chapter{Literature Review}\label{chapter:literature_review} 

The issues that this work aims to overcome are not new, and as such a lot of work has been performed
to address them. This section highlights the current state of the art in this domain, and where this
work fits within it.

\section{Spectral Element Methods}\label{section:literature_review:sem}

Spectral methods were developed by Gottlieb and Orszag~\cite{Gottlieb1977}, starting as early as
1969. Spectral methods attain high order accuracy by modeling solutions as sums of weighted
orthogonal functions. In 1984, Patera~\cite{Patera1984} proposed to combine spectral methods with
the finite-element method, creating the \acrfull{acr:SEM}. This method has the accuracy of spectral
methods and the geometric flexibility of finite-element methods.

\textit{Continuous} \acrlongpl{acr:SEM} force the solution to be continuous at the boundary between
elements by incorporating collocation points at the boundary of elements. \textit{Discontinuous}
\acrlongpl{acr:SEM} let the solution be discontinuous at element boundaries, and incorporate fluxes
to counterbalance these discontinuities. The discontinuous Galerkin formulation, first proposed by
Reed and Hill in 1973~\cite{Reed1973}, is combined with \acrlongpl{acr:SEM} to create the
\acrfull{acr:DG-SEM} by Warburton and Hesthaven~\cite{Hesthaven2002} in 2002. In 2009,
Kopriva~\cite{Kopriva2009} described a suite of algorithms that can be used to assemble a full
solver using this method. We will use the \acrshort{acr:DG-SEM} for its geometric flexibility and
the parallel simplicity of the flux formulation. This method is used in a wide array of domains,
including aerodynamics~\cite{Beck2014}, turbomachinery~\cite{Garai2015} and ocean and atmospheric
modeling~\cite{Gassner2016}.

Large scale spectral element solvers have existed for quite a while now. One such solver is Nek5000,
an open-source Navier Stokes spectral element solver that runs on \acrshortpl{acr:CPU} and uses
structured grids. It has been proven to scale well to above a million
processors~\cite{Offermans2017}. This solver has been used to perform simulations of micro-scale
blood flow~\cite{Obabko2017}, \acrfull{acr:DNS} of flows in internal combustion
engines~\cite{Ameen2020}, and nearly exascale simulations of flows in nuclear
reactors~\cite{Merzari2020}. Other solvers exist, such as the Nektar++~\cite{Cantwell2015} library.
It is a spectral finite element library used to create purpose-built solvers for different kinds of
problems, and works on unstructured grids. Example applications include aerodynamic
\acrfull{acr:LES} of sports cars~\cite{Mengaldo2020}, Formula One cars~\cite{Cantwell2015}, and
\acrshort{acr:DNS} of flows over various airfoils.

These are the kind of solid foundations we wish to build upon in this work. These solvers
traditionally use \acrshortpl{acr:CPU} to compute their solutions. The next logical step is to
perform these computations using \acrshortpl{acr:GPU} in order to increase their performance. These
methods will need to be ported to this new architecture. Additionally, none of these solvers use
\acrfull{acr:AMR}. This will have to be implemented in order to increase the efficiency of
computations, using the limited processing resources at our disposal where it counts most. This
creates the double challenge of using \acrshortpl{acr:GPU} for high-order methods and adaptivity,
which is especially challenging on \acrshortpl{acr:GPU} as they are more tailored for static
workloads. \Acrshortpl{acr:GPU} excel at solving very parallel problems where every core of the
\acrshort{acr:GPU} executes the exact same computation on neatly organised data. \Acrshort{acr:AMR}
is not a naturally parallel process, and the resulting meshes incur divergence in the execution on
different cores. Nonetheless, if \Acrshort{acr:AMR} is implemented correctly it should decrease the
execution time of the program while reducing resource usage.

\section{Graphics Processing Units}\label{section:literature_review:gpu}

In recent years, the scientific computing world has started using \acrshortpl{acr:GPU} to perform
computations. Some traditionally \acrshort{acr:CPU}-based commercial solvers were upgraded to run
entirely on \acrshortpl{acr:GPU}, or have some parts of the code \acrshort{acr:GPU}-accelerated.
Krawezik and Poole present a new \acrshort{acr:GPU}-accelerated direct sparse solver for Ansys
Fluent, boasting a \(2.9 \times \) speedup compared to the \acrshort{acr:CPU}-only solver.
Open-source solutions like OpenFOAM~\cite{Alonazi2015} have also been extended to work on hybrid
\acrshort{acr:CPU} and \acrshort{acr:GPU} architectures. Another example is
RapidCFD~\cite{SimFlow2020}, a collection of OpenFOAM solvers ported to Nvidia \acrshort{acr:CUDA}.
These solvers see an improvement of up to \(2 \times \) and a parallel speedup of \(3 \times \)
when using eight \acrshortpl{acr:GPU} instead of a single one.

Some brand new solvers have been created for \acrshortpl{acr:GPU}, such as the AmgX solver, a
distributed algebraic multigrid solver. This solver has been shown to have a \(2 \times \) to \(5
\times \) speedup compared to a \acrshort{acr:CPU} implementation~\cite{Naumov2015}.

In the realm of spectral methods, there have been a few advances using \acrshortpl{acr:GPU}. In
2009, Klöckner et al.\ demonstrated a \acrshort{acr:DG-SEM} solver running on a consumer level
\acrshort{acr:GPU} that demonstrated a \(50 \times \) speedup compared to the machine's
\acrshort{acr:CPU}~\cite{Klockner2009}. The next year, Gödel et al.\ presented a
\acrshort{acr:GPU}-accelerated discontinuous Galerkin solver for electric fields, capable of running
on multiple \acrshortpl{acr:GPU} in parallel~\cite{Godel2010}. The Nek5000 solver gained several
\acrshort{acr:GPU} compatible branches for Nekbone~\cite{Gong2016}\cite{Chalmers2022}, a simplified
version of the program that implements its core solver.

Purpose-built \acrshort{acr:GPU} solvers also exist. For example,  NekRS is a Navier Stokes solver
designed to run on diverse hardware, including \acrshortpl{acr:CPU} and
\acrshortpl{acr:GPU}~\cite{Fischer2021}. It is a rewrite of the Nek5000 solver, specifically for
parallel architectures. It uses the OCCA~\cite{Medina2014} library, a parallel \acrshort{acr:API}
that provides code that can be compiled to different architectures, notably OpenMP
\acrshort{acr:CPU} code, \acrshort{acr:CUDA} \acrshort{acr:GPU} code and OpenCL code. NekRS shows a
parallel efficiency of \(80 \% \), and \(80 \% \) to \(90 \% \) of the realizable peak performance
of \acrshort{acr:HPC} systems for large problems~\cite{Fischer2021}.

These works have shown that \acrshort{acr:SEM} solvers can have very good performance on the
\acrshort{acr:GPU} architecture. This gives a promising foundation for this work, as the core solver
needs to be fast. These solvers use static meshes however, meaning that the mesh does not change
during the simulation. This can be problematic for complex problems, when uniformly increasing the
mesh resolution to capture fine detail in the solution makes the problem prohibitively expensive to
compute. The mesh can be refined beforehand in areas of the flow expected to need refinement,
however these areas cannot always be predicted in advance, they may position in time, or may not be
easy to predict visually. The challenge of this work is to implement \acrlong{acr:AMR} in the
context of parallel \acrshort{acr:GPU} computing.

\section{Adaptive Mesh Refinement}\label{section:literature_review:amr}

\Acrshort{acr:AMR} has been shown to significantly speed up the computation of fluid flow problems.
Chalmers et al.\ report a \(3 \times \) speedup as well as an improved error when using
\acrshort{acr:AMR}~\cite{Chalmers2019}. The processing time and memory savings will be especially
welcome when considering that \acrshortpl{acr:GPU} have less memory than \acrshortpl{acr:CPU}.

\Acrshort{acr:AMR} implementations on \acrshortpl{acr:CPU} date from quite some time. Berger and
Oliger~\cite{Berger1984} presented the method using overset grids in 1984, and
Khokhlov~\cite{Khokhlov1998} described a tree-structured method that can be accessed and modified in
parallel in 1998.

Some software solutions exist to enhance existing code by implementing \acrlong{acr:AMR}. The
PARAMESH toolkit~\cite{MacNeice2000} for example has demonstrated speedups of over \( 10 \times \)
for high refinement levels, when compared to using a uniformly refined mesh. It also enables users
to work with the meshes it creates in parallel. The p4est library~\cite{Burstedde2011} provides
parallel \acrshort{acr:AMR} to programs by implementing forest of octrees algorithms. It has been
proven to scale well up to 220,000 \acrshort{acr:CPU} cores.

Traditional \acrshort{acr:CPU} \acrshort{acr:AMR} algorithms are well understood, and full books
have been written on the subject, such as the work by Plewa et al.~\cite{Plewa2005}. Support for
\acrshort{acr:AMR} has even started to be added to certain well-established solvers like
Nek5000~\cite{Offermans2019}, where the parallel scaling of the method has been studied up to 32,768
\acrshort{acr:CPU} cores~\cite{Peplinski2016}. This particular implementation uses h-refinement
only however.

These are good results, indicating that if the challenges in implementing \acrshort{acr:AMR} on
\acrshortpl{acr:GPU} are overcome, significant gains can be expected. These challenges are not
small, as the non-conforming meshes produced by \acrshort{acr:AMR} increase divergence between
\acrshort{acr:GPU} threads, and can create load imbalance when the problem is solved in parallel and
the mesh is not refined equally in each block. The refinement process itself is also not naturally
parallel, therefore care should be taken in the implementation to have as many parts of the process
executed in parallel on \acrshortpl{acr:GPU} as possible. Finally, adaptive meshes complicate the
memory layout on \acrshortpl{acr:GPU}, which are optimised to perform computations on static packed
data.

\Acrlong{acr:AMR} is not unique to fluid dynamics. The field of astrophysics has numerous
\acrshort{acr:AMR} solvers. First, Enzo~\cite{Bryan2014} is an open-source solver dedicated to
solving astrophysical fluid flows, notably dark matter dynamics and heating and cooling of flows.
This solver is \acrshort{acr:GPU}-accelerated, resulting in a \(5 \times \) speedup over its
\acrshort{acr:CPU}-only implementation. Then there are the GAMER~\cite{Schive2010} and
GAMER-2~\cite{Schive2018} solvers, both open-source \acrshort{acr:GPU}-accelerated
\acrshort{acr:AMR} solvers tailored to astrophysical simulations, including hydrodynamics,
magnetohydrodynamics, self-gravity and particle flows. The latter has been shown to scale up to 4096
\acrshortpl{acr:GPU} and 65,536 \acrshort{acr:CPU} cores. In the field of data visualisation, a lot
of work has to be done in order to directly represent the data generated by \acrshort{acr:AMR}
programs. Wang et al.~\cite{Wang2020} propose an algorithm to accurately render tree-based
\acrshort{acr:AMR} data in difficult regions such as across discontinuities in the refinement level.
Their method uses ray tracing and is computed on \acrshortpl{acr:CPU}, and can render data without
having to interpolate it to a single refinement level.

In the high-order fluid dynamics field, Giuliani and Krivodonova propose an algorithm for
\acrshort{acr:GPU}-based \acrlong{acr:AMR} for use with high-order methods to calculate gas
dynamics~\cite{Giuliani2019}. They use triangular meshes, and implement an edge-colouring algorithm
in order to avoid race conditions when computing fluxes. They show that \acrshort{acr:AMR} can be
used to solve problems that are not possible to solve in a reasonable amount of time without
refining the mesh locally, and that with a high order solver the time spent on \acrshort{acr:AMR}
can be relatively low, on the order of 4\%. This solver works on a single \acrshort{acr:GPU}, and as
such does not include dynamic load balancing to address the pitfalls of \acrshort{acr:AMR} on
multiple \acrshortpl{acr:GPU}.

We aim to implement \acrshort{acr:AMR} in our \acrshort{acr:DG-SEM} solver on \acrshortpl{acr:GPU}.
The \acrshort{acr:AMR} should be fast so that it does not dominate the execution time, therefore it
should be parallelised, run as much as possible on the \acrshort{acr:GPU}, and have a structure and
resulting meshes compatible with the \acrshort{acr:GPU} architecture. We target \acrshort{acr:HPC}
platforms, therefore the code should have an additional level of parallelism. The program should be
able to execute on multiple \acrshortpl{acr:GPU} and on multiple computers. This means that both the
solver and the \acrshort{acr:AMR} modules should be able to execute in parallel on multiple mesh
blocks. This adds an additional challenge, as the mesh is unlikely to be refined equally in all
\acrshortpl{acr:GPU}. This causes a load imbalance, and reduces the performance of the program as
the less heavily loaded \acrshortpl{acr:GPU} will have to wait for the most heavily loaded
\acrshort{acr:GPU} at each iteration. We implement dynamic load balancing in order to rearrange the
workload between the \acrshortpl{acr:GPU} to even out the computational load.

\section{Dynamic Load Balancing}\label{section:literature_review:load_balancing}

Load balancing is not uniquely a \acrshort{acr:CFD} issue. Many load balancing algorithms have been
developed in a variety of disciplines to address the issue. For example, Willebeek-LeMair and
Reeves~\cite{Willebeek1993} discuss different strategies to minimise the execution time of
applications running in parallel on different computers when the computational load cannot be
predicted. Cardellini et al.~\cite{Cardellini1999} describe load balancing algorithms for web
servers that avoid overwhelming servers and aim to fully utilise a cluster. In 1988, Kobayashi et
al.~\cite{Kobayashi1988} described an algorithm to dynamically load balance a parallel ray-tracing
system by subdividing the space into cubes, and dispatching them to a hierarchy of multiprocessors.
This approach has been shown to produce better results than static load balancing by \(30 \% \) to
\(50 \% \), and scale up to 1728 processors.

Implementing dynamic load balancing starts by choosing a repartitioning scheme. Graph-based
algorithms model a mesh as a group of nodes connected by edges, and try to partition the nodes while
breaking as few edges as possible. These solutions can be slow to execute, but result in
well-partitioned meshes. The METIS package~\cite{Karypis1997} implements such an algorithm. It
executes serially, partitioning meshes using multiple levels of graphs. The ParMETIS
library~\cite{Karypis1997P} implements a similar algorithm, and can work in parallel. A
repartitioning algorithm working in parallel is highly desirable for this work, as each
\acrshort{acr:GPU} is paired to a single \acrshort{acr:CPU} core. As much of the work as possible
will have to be executed in parallel on the \acrshort{acr:GPU}. These solvers can partition meshes
to multiple blocks and repartition \acrshort{acr:AMR} meshes~\cite{Karypis1997P}. These libraries
are widely used, notably by the Nek5000 \acrshort{acr:AMR} prototype~\cite{Peplinski2016}.

Another repartitioning scheme hinges on splitting a one-dimensional list of elements into smaller
segments, also called \acrfull{acr:CCP}. These algorithms are very fast, but the quality of their
partitions depend on the characteristics of the mapping from multidimensional space to the
one-dimensional list. One such mapping uses \acrfullpl{acr:SFC}. \Acrshortpl{acr:SFC} traverse a
multidimensional domain in its entirety, assigning a one-dimensional index to elements. The
Peano~\cite{Peano1890} and Hilbert~\cite{Hilbert1891} curves are examples of such curves.
\Acrshortpl{acr:SFC} are also widely used as repartitioning schemes, for example by the PARAMESH
library~\cite{MacNeice2000} and the GAMER family of solvers~\cite{Schive2018}.

In the domain of high-order methods, in 2021 He~\cite{He2021} created a \acrshort{acr:CPU}-based
dynamic load balancing algorithm for a \acrshort{acr:DG-SEM} solver with
\acrshort{acr:AMR}~\cite{He2021}, using the Hilbert curve backed by a hash map data structure. This
implementation showed good scaling to 16,384 \acrshort{acr:CPU} cores, and a speedup between \(5
\times \) and \(8 \times \) compared to a non load-balanced case.

Dynamically load balancing tasks executing on \acrshortpl{acr:GPU} is more complicated than on
traditional \acrshortpl{acr:CPU}. Due to the parallel execution of tasks on \acrshortpl{acr:GPU},
memory access patterns needed for optimal performance, and the higher cost of transferring data
between \acrshortpl{acr:GPU}, the performance may not scale as expected when redistributing a
workload. We want to leverage the increased processing power of \acrshortpl{acr:GPU}, therefore the
load balancing algorithm should not only redistribute a \acrshort{acr:GPU} workload, but also
execute on the \acrshortpl{acr:GPU} in parallel as much as possible. This very dynamic and serial
process will have to be adapted to the \acrshort{acr:GPU} architecture.

Dynamically load balancing tasks executing on \acrshortpl{acr:GPU} is an active area of research.
Chen et al.~\cite{Chen2010} discuss a task-based algorithm for balancing molecular dynamics
workloads on a single or multiple \acrshortpl{acr:GPU}. Their algorithm offers an up to \(2.5 \times
\) speedup, and scales linearly on up to four \acrshortpl{acr:GPU}. The load balancing itself
executes serially on the \acrshort{acr:CPU}. Kijsipongse and U-ruekolan~\cite{Kijsipongse2012}
propose an algorithm to load-balance K-Means clustering problems running on \acrshortpl{acr:GPU}
that outperforms a parallel \acrshort{acr:CPU} implementation by \(6.5 \times \).

Building on He's work~\cite{He2021}, we develop in this work a dynamic load balancing algorithm
based on \acrshort{acr:CCP} using the Hilbert curve, exploiting its speed of execution and good data
locality. The data locality quality of the curve is well suited for use on \acrshortpl{acr:GPU}, as
memory transfers resulting from large interfaces between \acrshortpl{acr:GPU} are even more
expensive than on \acrshortpl{acr:CPU}. The algorithm will execute in parallel for multiple
\acrshortpl{acr:GPU}, and aim to execute as much of the code as possible on the \acrshortpl{acr:GPU}
themselves to accelerate the process.
