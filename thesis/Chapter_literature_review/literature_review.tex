\chapter{Literature Review}\label{chapter:literature_review} 
% SEM
% Large scale SEM (Nek)
% Adaptivity
% GPUs
% Load balancing

The issues that this work aims to overcome are not new, and as such a lot of work has been perform
to overcome them. This section highlights the current state of the domain, and where this work fits
within it.

\section{Spectral Element Methods}\label{section:literature_review:sem}

Large scale spectral element solvers have existed for quite a while now. One such solver is Nek5000,
an open-source Navier Stokes spectral element solver that rune on \acrshortpl{acr:CPU}. It has been
proven to scale well to above a million processors~\cite{Offermans2017}. Other solutions exist, such
as the Nektar++~\cite{Cantwell2015} library. It is a spectral finite element library used to create
purpose-built solvers for different kinds of problems.

These are the kind of solid foundations we wish to build upon in this work. These solvers
traditionally use \acrshortpl{acr:CPU} to compute their solutions. We wish to use
\acrshortpl{acr:GPU}, so the methods will need to be ported to this new architecture.

\section{Graphics Processing Units}\label{section:literature_review:gpu}

In recent years, the scientific computing world has started using \acrshortpl{acr:GPU} to perform
computations. Some traditionally \acrshort{acr:CPU}-based commercial solvers were upgraded to run
entirely on \acrshortpl{acr:GPU}, or have some parts of the code \acrshort{acr:GPU}-accelerated.
Krawezik and Poole present a new \acrshort{acr:GPU}-accelerated direct sparse solver for Ansys
Fluent, boasting a \(2.9 \times \) speedup compared to the \acrshort{acr:CPU}-only solver.
Open-source solutions like OpenFOAM~\cite{Alonazi2015} have also been extended to work on hybrid
\acrshort{acr:CPU} and \acrshort{acr:GPU} architectures. Another example is
RapidCFD~\cite{SimFlow2020}, a collection of OpenFOAM solvers ported to Nvidia \acrshort{acr:CUDA}.

Some brand new solvers have been created for \acrshortpl{acr:GPU}, such as the AmgX solver, a
distributed algebraic multigrid solver. This solver has been shown to have a \(2 \times \) to \(5
\times \) speedup compared to a \acrshort{acr:CPU} implementation~\cite{Naumov2015}.

In the realm of spectral methods, there have been a few advances using \acrshortpl{acr:GPU}.
Notably, NekRS is a Navier Stokes solver designed to run on diverse hardware, including
\acrshortpl{acr:CPU} and \acrshortpl{acr:GPU}~\cite{Fischer2021}. It is related to the traditional
Nek5000 solver. It uses the OCCA~\cite{Medina2014}, a parallel programming language that provides
code that can be compiled to different architectures, notably OpenMP \acrshort{acr:CPU} code,
\acrshort{acr:CUDA} \acrshort{acr:GPU} code and OpenCL code. NekRS shows a parallel efficiency of
\(80 \% \), and \(80 \% \) to \(90 \% \) of the realizable peak performance of \acrshort{acr:HPC}
systems for large problems~\cite{Fischer2021}.

This indicates that \acrshort{acr:SEM} solvers can have very good performance on the
\acrshort{acr:GPU} architecture. This gives a promising foundation for this work, as the core solver
is expected to be fast. These solvers use static meshes, meaning that the mesh does not change
during the simulation. This can be problematic for complex problems, when uniformly increasing the
mesh resolution to capture fine detail in the solution makes the problem prohibitively expensive to
compute. The mesh can be refined beforehand in areas of the flow expected to need refinement, which
is not always possible if these areas cannot be predicted in advance, change position in time, or
are not easy to predict visually. We will therefore want to implement \acrlong{acr:AMR}, which is
not available in any of those well-known solvers.

\section{Adaptive Mesh Refinement}\label{section:literature_review:amr}

\Acrshort{acr:AMR} has been proved to speedup significantly 

% something about not being ideal for GPUs

~\cite{Schive2018} % GPU amr
~\cite{Giuliani2019} % GPU amr gas dynamics

\section{Dynamic Load Balancing}\label{section:literature_review:load_balancing}