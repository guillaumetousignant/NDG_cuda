\chapter{Discontinuous Galerkin Spectral Element Method} \label{chapter:spectral_element_method} 
We will use the discontinuous Galerkin spectral element method to solve this problem. This method is
part of the broader spectral methods, with elements from finite-element methods. This method has the
exponential convergence of spectral methods, and the ability to model difficult geometries of
finite-element methods. This method and a way to efficiently translate it to computer code is
described in~\cite{Kopriva2009}. We use this method to solve the 2D wave equation. The main problem
showcased through this work depicts a wave moving diagonally through a square domain.

\section{Spectral approximation} \label{section:spectral_element_method:spectral_approximation}
% Legendre polynomials, both forms
% Polynomial interpolation

Spectral methods, as described by Gotlieb and Orszag~\cite{Gottlieb1977}, model functions as sums of
weighed polynomials. $\phi_n(x)$ is the polynomial of degree $n$, and $a_n$ is the weight of said
polynomial. The weights make up the spectrum of the solution.

\begin{equation} \label{equ:infinite_sum}
    \begin{split}
        f(x) & = \sum_{n = 0}^{\infty }a_n \phi_n(x)
    \end{split}
\end{equation}

Since it is impractical to sum an infinite number of polynomials on a computer, the sum is truncated
to a polynomial order $N$. This becomes the polynomial order of the approximation. The truncated
part becomes $\tau$, the truncation error.

\begin{equation} \label{equ:2.1}
    \begin{split}
        f(x) & = \sum_{n = 0}^{\infty }a_n \phi_n(x) \\
        & = \sum_{n = 0}^{N}a_n \phi_n(x) + \sum_{n = N + 1}^{\infty }a_n \phi_n(x) \\
        & = \sum_{n = 0}^{N}a_n \phi_n(x) + \tau \\
        & \approx \sum_{n = 0}^{N}a_n \phi_n(x)
    \end{split}
\end{equation}

Since we discard part of the spectrum is discarded, it is important that the most important part of
the spectrum is in the lower order polynomials. We want the weights to decay as fast as possible.
This will be a characteristic of the polynomials we choose.

\subsection{Basis functions} \label{section:spectral_element_method:spectral_approximation:basis_functions}


\section{DG-SEM} \label{section:spectral_element_method:dg_sem}

~\cite{Patera1984}

\section{Equation} \label{section:spectral_element_method:equation}

\section{Implementation} \label{section:spectral_element_method:implementation}
% Talk about the flux computation, possible race conditions
