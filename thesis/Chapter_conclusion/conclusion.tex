\chapter{Conclusion}\label{chapter:conclusion}

\section{Summary}\label{section:conclusion:summary}

In this work, we presented a \acrshort{acr:GPU}-based wave equation solver using the
\acrlong{acr:DG-SEM}, as well as \acrlong{acr:AMR} and dynamic load balancing. The main
contributions of this work are: showing that \acrshort{acr:AMR} and load balancing on
\acrshortpl{acr:GPU} is possible and beneficial to solution quality and performance, showing that
significant performance can be gained by performing computations using \acrshortpl{acr:GPU}, and
showing that \acrshortpl{acr:GPU} accelerate spectral methods significantly.

The goal of using \acrshortpl{acr:GPU} to perform these computations was to increase the available
computing power, as spectral methods are notoriously expensive to compute. Spectral methods are
useful when highly accurate solutions of complex flows are needed.
Section~\ref{section:results:scaling_tests} shows that a baseline case without \acrshort{acr:AMR}
can be computed on a \acrshort{acr:HPC} platform up to three times faster using \acrshortpl{acr:GPU}
when they are sufficiently loaded. The program has also been shown to scale well up to \(64\)
\acrshortpl{acr:GPU}. Chapter~\ref{chapter:graphics_processing_units} discusses the architectural
decisions taken to use the massively parallel \acrshort{acr:GPU} architecture. One such decision is
to implement a data structure such that only high-level objects such as elements and faces are
stored in flat arrays that can be transferred from the \acrshort{acr:GPU} to the \acrshort{acr:CPU}.
The data arrays within those objects are instead allocated directly by the \acrshort{acr:GPU} in
\acrshort{acr:GPU} code, and stored in dynamic memory. This allows for a more flexible data
structure, where the \acrshort{acr:GPU} itself can perform some of the dynamic mesh changes
necessary for \acrshort{acr:AMR} and load balancing, and reduces the overhead of moving objects
around because the bulk of the data is stored outside of the objects themselves.

Some problems have localised areas where more precision is needed, such as shocks and boundary
layers. Even with the increased processing power of \acrshortpl{acr:GPU}, uniformly refining a mesh
to the level needed by those areas makes the computing time increase to the point where it is not
economic to solve these problems. We use \acrlong{acr:AMR} to assess which parts of the problem have
a higher level of estimated error and to refine them, all while the program is running.
Section~\ref{section:results:adaptivity_performance} shows that a case with \acrshort{acr:AMR} can
be more than \(67 \times \) faster to execute than the same case initially uniformly refined to the
same level as the \acrshort{acr:AMR} case. When compared to another case initially uniformly refined
to the point that it reaches a similar maximum error to the \acrshort{acr:AMR} case, the
\acrshort{acr:AMR} case is still generally faster. This is an excellent result, given that
\acrshortpl{acr:GPU} are not tailored to these kinds of dynamically changing computations. 

The \acrshort{acr:AMR} process itself does not match well with the \acrshort{acr:GPU} architecture,
as it reallocates and moves large amounts of memory, and is fundamentally sequential. Since each
\acrshort{acr:GPU} is paired with a single \acrshort{acr:CPU} and is vastly faster, we want to
execute as much as possible of the process in parallel on the \acrshort{acr:GPU}. We devised an
algorithm to perform \acrshort{acr:AMR} in parallel, described in
Chapter~\ref{chapter:adaptive_mesh_refinement}, offloading most of the computation to the
\acrshort{acr:GPU}. Moving elements, h-refining, p-refining and the renumbering that comes with the
process are all executed on the \acrshort{acr:GPU}. The only significant parts executed on the
\acrshort{acr:CPU} are the allocation of the resized high-level arrays and the computing of offset
arrays, so that the different \acrshort{acr:GPU} threads can operate in parallel without race
conditions. Despite this, \acrshort{acr:AMR} can be a costly process on \acrshortpl{acr:GPU}, for
example in Subsection~\ref{subsection:results:complex_meshes:profiling} where it makes up a
significant portion of the total runtime.

We also implement a mesh pre-condition algorithm, described in
Section~\ref{section:adaptive_mesh_refinement:pre_conditioning}, to refine the mesh before the
computation up to a point where initial conditions are well resolved.
Section~\ref{section:results:adaptivity_performance} shows that performing a few pre-condition steps
can significantly increase the solution quality when the starting mesh is very coarse. Combined with
\acrshort{acr:AMR}, this means that meshes do not need to be very tailored to problems. A uniform
coarse mesh can be used, and the pre-condition will refine the mesh to capture initial conditions
correctly, while \acrshort{acr:AMR} refines the mesh as the solution is computed to capture the
important areas of the flow.

To combat the load imbalance that can arise when the problem is solved in parallel using multiple
\acrshortpl{acr:GPU}, we implemented a dynamic load balancing algorithm. The algorithm uses the
Hilbert curve, a \acrlong{acr:SFC} that has good locality. This is especially important when using
\acrshortpl{acr:GPU} because data transfers between \acrshortpl{acr:GPU} are more expensive,
therefore we want to reduce the size of the boundaries between mesh blocks as much as possible.
Section~\ref{section:results:load_balancing_performance} shows that dynamic load balancing
significantly increases performance when there is load imbalance. The performance increase scales
well with load imbalance, meaning that dynamic load balancing should improve the performance no
matter how much load imbalance is present. By judiciously choosing when we load balance, as
explained in Section~\ref{section:load_balancing:criteria}, the computing time spent in the load
balancing algorithm can be very low, as reported in
Subsection~\ref{subsection:results:complex_meshes:profiling}. 

Load balancing is hampered  by the same limitations as \acrshort{acr:AMR}, namely that it is not
particularly well suited for the \acrshort{acr:GPU} architecture, and we want to execute as much of
it as possible on the \acrshort{acr:GPU}. \Acrshortpl{acr:GPU} have less memory than
\acrshortpl{acr:CPU}, therefore we want each worker \acrshort{acr:GPU} to only have knowledge about
the mesh block it is assigned in order to reduce its memory footprint. This complicates load
balancing, as it is harder to reconstruct the mesh once elements are exchanged. Because of this, the
load balancing algorithm is made up of several smaller dependent functions, in order to catch every
possible edge case. Most of those functions execute on the \acrshort{acr:GPU}, including the
generation of the Hilbert curve ordering of the elements. This algorithm has a good performance,
despite performing many data transfers between \acrshortpl{acr:GPU} and reallocating a lot of
memory. Chapter~\ref{chapter:load_balancing} details how the algorithm is designed, and what
strategies were used to make it perform well on \acrshortpl{acr:GPU}.

Another significant conclusion is that \acrshortpl{acr:GPU} are particularly well suited to the
spectral methods, specifically. As described in Chapter~\ref{chapter:spectral_element_method}, the
number of operations to perform every time steps scales with \({\left( N + 1 \right)}^2\), where
\(N\) is the polynomial order of an element. This indicates that the computational complexity should
scale with \({\left( N + 1 \right)}^2\). However, as reported in
Subsection~\ref{subsection:results:load_balancing_performance:polynomial_order}, it seems that
increasing the polynomial order of elements increases the density of computations, which lends
itself to better performance. The computation time relative to \(N\) seems closer to a linear
relation than a quadratic one. This means that using higher order elements, with the benefit of
higher accuracy, is more attractive on \acrshortpl{acr:GPU} than on traditional platforms. It also
points to the fact that spectral methods in general could have an increased speedup when using
\acrshortpl{acr:GPU} compared to other methods.

\section{Concluding Remarks}\label{section:conclusion:remarks}

The \acrshort{acr:AMR} and load balancing algorithms destined for \acrshortpl{acr:GPU} are very
different from those destined for \acrshortpl{acr:CPU} only. Many different strategies must be used,
such as optimising each and every part of the algorithm in terms of what can be executed in parallel
on shared memory and what cannot. These parts should be executed on the \acrshort{acr:GPU} in order
to improve performance and avoid unnecessary copies from the \acrshort{acr:GPU} to the
\acrshort{acr:CPU}. Once these parallel parts have been identified, they must be modified until they
operate on a single object type. For example, a kernel that is parallelised over elements and moves
elements to a new index can modify the elements, arrays that are indexed per element, but it cannot
modify faces. This is because multiple elements can link to the same face. If for example that
kernel would update the element indices in the faces, a race condition would occur and the results
would not be correct. In this case, the faces should update their element indices in another kernel,
that parallelises over faces. There are many such issues that do not impact sequential code but
start appearing when a few thousand threads execute in parallel.

Another noteworthy difference about \acrshort{acr:GPU} computing is that surprising things can have
an impact on performance. One example is increasing the thread block size in
Subsection~\ref{subsection:results:scaling_tests:strong}. Increasing the block size should improve
performance by reducing the number of instructions to dispatch. This is because there is a lot of
divergence in the code, and with an increased block size more threads will be waiting for divergents
threads. It is important to profile the code and look at which parts are taking up more time than
they should. Sometimes some problematic functions can be made much faster, such as the error
estimation routine that went from taking up \(15 \% \) of the total execution time to \(0.1 \% \) by
precomputing a set of values.

The \acrshortpl{acr:GPU} need to be saturated with work in order to have good performance. This can
lead to peculiar situations like those in Subsection~\ref{subsection:results:scaling_tests:strong},
where the relation between number of elements in a \acrshort{acr:GPU} and the computation time is
not linear. Under a certain workload, the cores of a \acrshort{acr:GPU} do not all have work to
perform, and the \acrshort{acr:GPU} is partly idle. On the other hand, if there is too much work in
a \acrshort{acr:GPU} its memory may fill up, or the constant cache swapping from each core working
on different elements successively may degrade performance. There is an ideal amount of work per
\acrshort{acr:GPU} to be found, depending on the problem.

Thread divergence can also cause surprising results, such as a single if/else statement reducing the
performance of a function by half. For example, the projection from faces to elements has four
different paths, depending on if the face is forward or backward compared to the element, and if it
is conforming or not. Since threads execute in groups of 32, threads that do not take a particular
branch are stopped while the threads that took that branch execute, and then the next branch is
evaluated. For our projection, if it happens that there are threads that take each of the four
branches in a group, each of the branches will be executed one after the other with different
threads inactive, taking up four times as much time as it should. This is a significant difference
when programming for \acrshortpl{acr:GPU}: branching must be minimised as much as possible.

It is also difficult to program dynamic meshes on the \acrshort{acr:GPU}. The fact that neither the
\acrshort{acr:CPU} or \acrshort{acr:GPU} have the complete picture, and that all data has to be
transferred from one to the other complicates some of the algorithms. Load balancing in particular
required the most changes compared to a sequential \acrshort{acr:CPU} algorithm. In that algorithm,
there are many dependencies between the different parts, which must be computed in a specific order.
It is possible that it would be easier to re-create the mesh once elements have been exchanged, and
rebuild all the connectivity from scratch. That approach was not chosen; instead the mesh is
modified to remove unneeded parts and add new parts. Seeing that modifying the mesh with limited
information is very difficult and error-prone, it is hard at this stage to say which approach is
best.

\section{Future Work}\label{section:conclusion:future_work}

\subsection{GPU Computing}\label{subsection:conclusion:future_work:gpu}

There is still a lot of work to be done in order to better match the \acrshort{acr:GPU} architecture
to the different modules of the program. For one, there is a lot of divergence in the code, due to
\acrshort{acr:AMR} creating elements with different polynomial orders and non-conforming interfaces
among other things. An avenue for future work would be to try to reduce divergence as much as
possible. For example, sorting the faces by type and storing them in different arrays would allow
different kernels to be launched for each type of face. In that case, each \acrshort{acr:GPU} thread
in a kernel would execute the same instructions and there would be no divergence. Similarly,
refining elements per block would alleviate divergence, in that the elements assigned to a block of
threads are always kept the same and there is no divergence.

Another approach would be to decompose the objects, like faces and elements, to a separate array for
each member. This means changing from an ``array of structures'' to a ``structure of arrays'' data
structure. This could help alleviate cache pressure, since \acrshortpl{acr:GPU} have smaller caches
than \acrshortpl{acr:CPU}. This helps by reducing the amount of data that has to be loaded in cache
since only the data members used can be loaded instead of whole objects. This could improve
performance, especially when the \acrshort{acr:GPU} is very loaded.

There is also always work to be done by profiling the code and examining which parts of the code
take up more execution time than they should, or are not optimal. Many tools exist to profile
\acrshort{acr:CUDA} code. Two profilers were used to summarily examine the code for this work. The
first is an overall profiler, the result of which is shown in
Subsection~\ref{subsection:results:complex_meshes:profiling}. The other is a profiler for single
kernels, which was used to improve the error estimation kernel. These tools give a lot of
information, such as latency statistics, data dependencies, and optimal occupancy of the
\acrshort{acr:GPU}.

There is no reason to use only \acrshortpl{acr:CPU} or only \acrshortpl{acr:GPU}. A future path to
explore is to make a hybrid solver. If the program was modified to enable using both
\acrshort{acr:CPU} and \acrshort{acr:GPU} workers together, it would be possible to use the complete
processing power of computers. Since \acrshortpl{acr:CPU} and \acrshortpl{acr:GPU} have very
different computing powers and capacities, this would make load balancing more difficult as elements
would need to be split unequally between the workers. Since we implemented a \acrshort{acr:CPU}
version of the program as a comparison, and both versions have the same interface between workers,
it was possible to create a crude hybrid version to assess its potential. This tentative hybrid
version is presented in Appendix~\ref{chapter:hybrid_solver}.

Finally, another area of work would be to add more fine-grained parallelism. As it stands, every
process communicates with others via \acrshort{acr:MPI}. This happens regardless of if
the other process is on the same machine or not. Workers on the same machine could access the same
memory, and even transfer data between \acrshortpl{acr:GPU} directly if they are connected together.
We could use multithreading in addition to multiprocessing, where there is one process per machine
and all workers on a machine execute via threads. This would completely remove necessary transfers
between workers on the same machine, as they could all access the same data. This would also reduce
the number of \acrshort{acr:MPI} transfers needed, as only one process per machine would need to
communicate.

\subsection{DG-SEM}\label{subsection:conclusion:future_work:dg_sem}

In this work, we solved the wave equation. We use this simple equation in order to show the program
works, but it is not very useful in itself. A possible avenue would be to upgrade the solver to
solve a more interesting and complex equation. For example, solving the full Navier-Stokes equations
would make the program useful for solving real-world problems.

Another possible improvement is using different shapes other than quadrilaterals, ideally within the
same mesh. Many meshes use triangles, and some unstructured meshes use a mix of quadrilaterals and
triangles. Some meshes even use elements with an arbitrary number of sides. Supporting these element
types would make the program more general and able to work with more off-the-shelf meshes. This
would mean changing the structure of the program to be generic over element types.

Speaking of elements, we often want to model curved surfaces, such as the airfoil from
Subsection~\ref{section:results:complex_meshes}. In order to model curved surfaces with
straight-edged elements, the elements need to be very small, and only approximate such surfaces.
Implementing curved elements would help model these geometries better. An added advantage of curved
elements is the possibility of smoothing the transition between very skewed elements, which becomes
a problem with more complex equations.

Finally, it would be very interesting to transform the program from a 2D solver to a 3D solver. Many
interesting problems appear only in 3D, and more complex geometries can be modeled. Also, 3D
problems are an order of magnitude more computationally intensive to solve. These problems would
benefit even more from the computational power of \acrshortpl{acr:GPU} and would have no problem
saturating them with work thanks to the increased complexity. With another dimension to fill, the
number of elements that is possible to put in each dimension is reduced. This means that
\acrshort{acr:AMR} is even more crucial in 3D, as computational resources are limited. The
computational complexity increases from \({\left( N + 1 \right)}^2\) to \({\left( N + 1 \right)}^3\)
in 3D, where \(N\) is the polynomial order. The scaling shown in
Subsection~\ref{subsection:results:load_balancing_performance:polynomial_order} could help with that
increased complexity.

\subsection{AMR}\label{subsection:conclusion:future_work:amr}

There is also work to do to improve \acrlong{acr:AMR}. Firstly, it takes up a large proportion of
the total computation time. Maybe being more judicious about when we refine could alleviate this
problem. There could be a global target error estimate below which the \acrshort{acr:AMR} routine is
not launched, similar to the load balancing threshold described in
Section~\ref{section:load_balancing:criteria}. Care should be taken to not compromise efficiency by
refining more often than necessary. It may also be possible to refine only some parts of the mesh,
and avoid moving and reallocating the whole mesh.

As it stands now, we do not perform mesh coarsening in this work. This is an important avenue to
explore, as it saves even more resources than refinement alone. In many cases, such as the one shown
in Section~\ref{section:results:test_case}, the areas that need refinement are not static in space.
In that example, a wave goes through the domain diagonally. As it advances, the mesh is refined to
better capture the wave. In its wake are left many very refined elements that do not contribute to
the solution anymore, as the wave is not present in those areas. These elements take up computation
power for no benefit. Mesh coarsening would do the inverse of h-refinement and p-refinement, and
either lower the polynomial order of elements or merge elements together. This would be a
significant challenge to add to the program because of its flat arrays data structure. Many
\acrshort{acr:AMR} programs store their elements in a tree data structure in order to keep track of
which elements are the children of which and to be able to re-form elements that have split. In our
data structure, elements resulting of h-refinement are just regular elements like any other, and
have no knowledge of their previous topology. Coarsening would imply searching through nearby
elements for other elements to be coarsened that can form a quadrilateral together. This is
difficult, but brings the interesting ability of re-forming elements that did not exist previously.

\subsection{Dynamic Load Balancing}\label{subsection:conclusion:future_work:load_balancing}

The implementation of dynamic load balancing we present operates on the assumption that all elements
have the same weight and are all equally as expensive to compute. With \acrshort{acr:AMR}, all
elements can have a different polynomial order. As shown in
Subsection~\ref{subsection:results:load_balancing_performance:polynomial_order}, the polynomial
order of elements influences the computation time. A way to load balance more fairly would be to
give a weight to each element, and split that weight equally between workers. That weight could be
influenced by an element's polynomial order and its number of neighbours. This is not trivial to
add, as then all workers would need to be aware of the weight of all elements in order to know how
to split the workload equally.

Since we would like to add a hybrid solver as stated in
Subsection~\ref{subsection:conclusion:future_work:gpu} and assign weights to elements of different
complexity, the next logical step is to assign different capacities to workers. This is evident in
the case of a hybrid solver, as a \acrshort{acr:GPU} has vastly more computing power than a
\acrshort{acr:CPU}. By assigning a different capacity to each worker, \acrshort{acr:GPU} workers can
be assigned more work than \acrshort{acr:CPU} workers. Even in the case of a \acrshort{acr:GPU} only
solver, assigning capacity to workers would enable better load balancing on heterogenous platforms:
for example, different computers with different kinds of \acrshortpl{acr:GPU} working together, or
full \acrshortpl{acr:GPU} working with fractions of \acrshortpl{acr:GPU} such as when sharing a
system. 

This capacity per worker could even be computed on the fly, with the program performing some
iterations, and then examining how much time each worker took to perform such iterations. If we
correlate time and the number of elements in each worker, the capacity of the workers could
be computed. This would enable a hybrid solver to work well out of the box without doing trial runs
or guessing the relative weight of specific \acrshortpl{acr:CPU} and \acrshortpl{acr:GPU}. This
could even tailor the capacity of identical \acrshortpl{acr:GPU} to unit-to-unit differences, or
alleviate the load on thermally limited \acrshortpl{acr:GPU} in systems where heat is an issue.
