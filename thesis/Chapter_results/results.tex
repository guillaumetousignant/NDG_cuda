\chapter{Results} \label{chapter:results}
% Setup, supercomputers etc
% Profiler
% Memory usage maybe?
% Stream performance
% Mixed precision maybe?
% Hybrid solver?
% Different block sizes!!

\section{Test case} \label{section:results:test_case}
% new case

\section{Scaling tests} \label{section:results:scaling_tests}
% Scaling tests (strong and weak), N tests
% CPU vs GPU
% GPU loading

\subsection{Strong scaling} \label{section:results:scaling_tests:strong}

\begin{figure}[H]
	\centering
	\includesvg[width=0.6\textwidth]{Chapter_results/media/strong_scaling_N4_K65536_W32}
	\caption{Weak scaling: The problem size increases with the number of workers. N = 4, K = 65536, W = 32}
	\label{fig:strong_scaling_N4_W32}
\end{figure}

\begin{figure}[H]
	\centering
	\includesvg[width=0.6\textwidth]{Chapter_results/media/strong_scaling_N6_K65536_W32}
	\caption{Weak scaling: The problem size increases with the number of workers. N = 6, K = 65536, W = 32}
	\label{fig:strong_scaling_N6_W32}
\end{figure}

\begin{figure}[H]
	\centering
	\includesvg[width=0.6\textwidth]{Chapter_results/media/strong_scaling_N6_K65536_W64}
	\caption{Weak scaling: The problem size increases with the number of workers. N = 6, K = 65536, W = 64}
	\label{fig:strong_scaling_N6_W64}
\end{figure}

\begin{figure}[H]
	\centering
	\includesvg[width=0.6\textwidth]{Chapter_results/media/strong_scaling_N6_K65536_W128}
	\caption{Weak scaling: The problem size increases with the number of workers. N = 6, K = 65536, W = 128}
	\label{fig:strong_scaling_N6_W128}
\end{figure}

\subsection{Weak scaling} \label{section:results:scaling_tests:weak}


\begin{figure}[H]
	\centering
	\includesvg[width=0.6\textwidth]{Chapter_results/media/weak_scaling_N4_K4096_W32}
	\caption{Weak scaling: The problem size increases with the number of workers. N = 4, K = 4096/worker, W = 32}
	\label{fig:weak_scaling}
\end{figure}

\section{Adaptivity performance} \label{section:results:adaptivity_performance}
% Error adaptive vs non-adaptive, time to run
% Error at same runtime

\begin{figure}[H]
	\centering
	\includesvg[width=0.6\textwidth]{Chapter_results/media/adaptivity_N4_K16_C5}
	\caption{Adaptivity efficiency: Simulation time and analytical error with adaptivity and increasing pre-processing steps. N = 4, K = 16, adaptivity interval = 5}
	\label{fig:adaptivity_efficiency_C5}
\end{figure}

\begin{figure}[H]
	\centering
	\includesvg[width=0.6\textwidth]{Chapter_results/media/adaptivity_N4_K16_C20}
	\caption{Adaptivity efficiency: Simulation time and analytical error with adaptivity and increasing pre-processing steps. N = 4, K = 16, adaptivity interval = 20}
	\label{fig:adaptivity_efficiency_C20}
\end{figure}

\begin{figure}[H]
	\centering
	\includesvg[width=0.6\textwidth]{Chapter_results/media/adaptivity_N4_K16_C100}
	\caption{Adaptivity efficiency: Simulation time and analytical error with adaptivity and increasing pre-processing steps. N = 4, K = 16, adaptivity interval = 100}
	\label{fig:adaptivity_efficiency_C100}
\end{figure}

\begin{figure}[H]
	\centering
	\includesvg[width=0.6\textwidth]{Chapter_results/media/adaptivity_N4_K16_C500}
	\caption{Adaptivity efficiency: Simulation time and analytical error with adaptivity and increasing pre-processing steps. N = 4, K = 16, adaptivity interval = 500}
	\label{fig:adaptivity_efficiency_C500}
\end{figure}

\section{Load balancing performance} \label{section:results:load_balancing_performance}
% Runtime adaptive with and without load balancing

% Show the problem

\subsection{Load balancing interval} \label{section:results:load_balancing_performance:interval}

\begin{figure}[H]
	\centering
	\includesvg[width=0.6\textwidth]{Chapter_results/media/load_balancing_interval_N4_K16384_A20_P16_S3}
	\caption{Load balancing efficiency interval test: Simulation time with adaptivity, load balancing and increasing load balancing interval. N = 4, K = 16384, adaptivity interval = 20, P = 16, max split level = 3}
	\label{fig:load_balancing_efficiency_interval}
\end{figure}

\subsection{Load balancing threshold} \label{section:results:load_balancing_performance:threshold}

\begin{figure}[H]
	\centering
	\includesvg[width=0.6\textwidth]{Chapter_results/media/load_balancing_threshold_N4_K16384_A20_L20_P16_S3}
	\caption{Load balancing efficiency threshold test: Simulation time with adaptivity, load balancing and increasing load balancing threshold. N = 4, K = 16384, adaptivity interval = 20, load balancing interval = 20, P = 16, max split level = 3}
	\label{fig:load_balancing_efficiency_threshold}
\end{figure}

\section{Complex meshes} \label{section:results:complex_meshes}
